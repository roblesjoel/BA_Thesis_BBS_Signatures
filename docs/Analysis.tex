%============================ Analysis Document ================================
% define document class
\documentclass[
	a4paper               % paper format
%	,10.5pt               % fontsize
%	,BCOR=18mm            % Binding correction
	,bibliography=totoc   % If enabled add bibliography to TOC
	,listof=totoc         % If enabled add lists to TOC
%	,bilingual
	,monolingual
]{bfhthesis}              % KOMA-script report

\usepackage[
	hidelinks,
	pdfusetitle,
]{hyperref}
\usepackage{amsmath}
\usepackage{listings}

\LoadBFHModule{boxes}

\colorlet{punct}{red!60!black}
\definecolor{background}{HTML}{EEEEEE}
\definecolor{delim}{RGB}{20,105,176}
\colorlet{numb}{magenta!60!black}

\lstdefinelanguage{json}{
    basicstyle=\normalfont\ttfamily,
    numbers=left,
    numberstyle=\scriptsize,
    stepnumber=1,
    numbersep=8pt,
    showstringspaces=false,
    breaklines=true,
    frame=lines,
	postbreak=\mbox{\textcolor{red}{$\hookrightarrow$}\space},
    backgroundcolor=\color{background},
    literate=
      {:}{{{\color{punct}{:}}}}{1}
      {,}{{{\color{punct}{,}}}}{1}
      {\{}{{{\color{delim}{\{}}}}{1}
      {\}}{{{\color{delim}{\}}}}}{1}
      {[}{{{\color{delim}{[}}}}{1}
      {]}{{{\color{delim}{]}}}}{1},
}

\lstdefinelanguage{canon}{
    basicstyle=\normalfont\ttfamily,
    numbers=left,
    numberstyle=\scriptsize,
    stepnumber=1,
    numbersep=8pt,
    showstringspaces=false,
    breaklines=true,
    frame=lines,
    backgroundcolor=\color{background},
	postbreak=\mbox{\textcolor{red}{$\hookrightarrow$}\space},
}

\hyphenation{ve-ri-fi-ca-ti-on}

\begin{document}

\frontmatter

\title{Bachelor's Thesis}
\subtitle{Unlinkability of Verfiable Credentials in a practical approach: Analysis Part}
\author{Joël Gabriel Robles Gasser}
\institution{Bern University of Applied Sciences}
\department{Engineering and Computer Science}
\institute{Computer Science}
\version{0.1}
\advisor{Prof. Dr. Annett Laube \and Prof. Dr. Reto Koenig}
\expert{Dr. Andreas Spichiger}
\degreeprogram{Bachelor of Science in Computer Science}

%----------------  BFH tile page   -----------------------------------------
\maketitle

\addchap{Abstract}
Here an abstract might be placed.


%------------ TABLEOFCONTENTS ----------------
\tableofcontents

%------------ START MAIN PART ----------------
\mainmatter

\chapter{How to prepare VCs for BBS}
\label{chap:vctobbs}

\section{Shuffled label map}
\label{sec:shuffledlabelmap}

Because of security considerations mentioned in chapter 6.1.5.1 of the Project Management document the canonicalised \textbf{vc} needs to be shuffled.\\

The required inputs of this algorithm is random 32-bit long bit-string named \textbf{hmac\_key} which will be used to initialize a \textit{HMAC}.\\

The output is a function called \textbf{label\_map\_factory\_function}.\\

We follow these steps:
\begin{itemize}
	\item Create a function with the name \textbf{label\_map\_factory\_function}, which has one required input named \textbf{canonical\_id\_map}. Inside the function do these steps:
	\begin{itemize}
		\item Initialize \textbf{bnode\_id\_map} to a new map.
		\item For each entry in the \textbf{canonical\_id\_map}, which are split up into key - value
		\begin{itemize}
			\item Set \textbf{hmac\_result} to result of a HMAC-digest operation, which was initialized with the \textbf{hmac\_key}, on the value of the entry. 
			\item Add a new map entry into \textbf{bnode\_id\_map}, where the key is the key from \textbf{canonical\_id\_map} and the value is \textbf{hmac\_result} encoded in base64-url encoding defined in chapter 5 and 3.2 of "RFC 4648"\cite{base64}.
		\end{itemize}
		\item Initialize \textbf{hmac\_ids} to the sorted values of \textbf{bnode\_id\_map}
		\item For each key in the \textbf{bnode\_id\_map}
		\begin{itemize}
			\item Set the value for the given key in \textbf{bnode\_id\_map} with \textbf{b} concatenated with the index of the correspoing value in \textbf{hmac\_ids}. The value should then look like \textbf{b0}.
		\end{itemize}
		\item Finally return \textbf{bnode\_id\_map}.
	\end{itemize}
	\item Return \textbf{label\_map\_factory\_function}
\end{itemize}

\section{Create the proof config}
\label{subsubsec:proof_config}
The proof config contains the information about the proof to be generated, like what cryptosuite was used, what type of proof it is etc.\\

The required inputs of this algorithm are the url of the verification information \textbf{verification\_method} and the context of the VC \textbf{@context}.\\

The output of this algorithm is the proof config \textbf{proof\_config} and the canonicalised proof config \textbf{canonical\_proof\_config}.\\

We follow these steps:
\begin{itemize}
	\item Create an empty \textbf{proof\_config} object
	\item Set \textbf{proof\_config.type} to \textit{DataIntegrityProof}
	\item Set \textbf{proof\_config.cryptosuite} to \textit{bbs-2023}
	\item Set \textbf{proof\_config.created} to the current time
	\item Set \textbf{proof\_config.verificationMethod} to the url of the verification information \textbf{verification\_method} (see: \url{https://www.w3.org/TR/vc-di-bbs/#multikey})
	\item Set \textbf{proof\_config.@context} to \textbf{@context}
	\item Set \textbf{canonical\_proof\_config} to the canonical representation of the \textbf{proof\_config} using the \textit{Universal RDF Dataset Canonicalization Algorithm}\cite{rdf}
	\item Return an object containing \textbf{proof\_config} and \textbf{canonical\_proof\_config}.
\end{itemize}

If we follow the instructions step by step we get a JSON object like this:
\begin{lstlisting}[language=json,firstnumber=1,caption={Example Proof Config},captionpos=b]
{
	"type": "DataIntegrityProof",
	"cryptosuite": "bbs-2023",
	"created": "2024-04-01T21:28:52.401Z",
	"verificationMethod": "www.example.com/keys",
	"proofPurpose": "assertionMethod",
	"@context": [
		"https://www.w3.org/ns/credentials/v2"
	],
}
\end{lstlisting}\newpage

Which in turn is canonicalised into this:

\begin{lstlisting}[language=canon,firstnumber=1,caption={Example Proof Config canonicalized},captionpos=b]
_:c14n0 <http://purl.org/dc/terms/created> "2024-04-01T21:35:10.679Z"^^<http://www.w3.org/2001/XMLSchema\#dateTime> .
_:c14n0 <http://www.w3.org/1999/02/22-rdf-syntax-ns\#type> <https://w3id.org/security\#DataIntegrityProof> .
_:c14n0 <https://w3id.org/security\#cryptosuite> "bbs-2023"^^<https://w3id.org/security\#cryptosuiteString> .
_:c14n0 <https://w3id.org/security\#proofPurpose> <https://w3id.org/security\#assertionMethod> .
_:c14n0 <https://w3id.org/security\#verificationMethod> www.example.com/keys .
\end{lstlisting}

With that we have our \textbf{canonical\_proof\_config} and the first step is finished.

\section{Transform the VC}
\label{chap:transform}
This algorithm transforms VC into statements which can later be signed by BBS.\\

The required inputs are a VC document \textbf{vc}, a 32-byte long \textbf{hmac\_key} to initialize a HMAC and a set of mandatory pointers \textbf{mandatory\_attributes}.\\

We follow these steps:
\begin{itemize}
	\item Define \textbf{label\_map\_factory\_function} as the function which is returned from the \textbf{create\_shuffled\_id\_label\_map\_function} defined in chapter \ref{sec:shuffledlabelmap} with the input \textbf{hmac\_key}.
	\item Set \textbf{group\_definitions} to a map where \textit{mandatory} is the key and the value is \textbf{mandatory\_attributes}.
	\item Set \textbf{groups} to the \textit{response.groups} of the \textit{canonicalizeAndGroup} algorithm specified in the DataIntegrity for ECDSA specification\cite{ecdsa}, passing the \textbf{label\_map\_factory\_function}, \textbf{group\_definitions} and \textbf{vc}.
	\item Set \textbf{mandatory} to the value from \textbf{groups.mandatory.matching}
	\item Set \textbf{nonMandatory} to the value from \textbf{groups.mandatory.nonMatching}
	\item Return an object named \textbf{transformed\_data} containing \textit{mandatory} set to \textbf{mandatory}, \textit{nonMandatory} set to \textbf{nonMandatory} and \textit{hmacKey} set to \textbf{hmacKey}.
\end{itemize}\newpage

This object may look like this:

\begin{lstlisting}[language=json,firstnumber=1,caption={Return object of the VC transformation},captionpos=b]
{
	"mandatory": [[
		0,
		"_:b0 <https://www.w3.org/ns/credentials/issuer-dependent#first_name> \"Joel\" .\n",
		],
		[
		2,
		"_:b1 <http://www.w3.org/1999/02/22-rdf-syntax-ns#type> <https://www.w3.org/2018/credentials#VerifiableCredential> .\n",
		],
		[
		3,
		"_:b1 <https://www.w3.org/2018/credentials#credentialSubject> _:b0 .\n",
		]],
	"nonMandatory": [[
		1,
		"_:b0 <https://www.w3.org/ns/credentials/issuer-dependent#last_name> \"Robles\" .\n",
	  ]],
	"hmacKey": [0, 17, ...],
}
\end{lstlisting}

And with that we transformed the VC into canonical statements that can later be signed by BBS.

\section{Hash the proof data}
\label{subsubsec:proofHash}
This algorithm creates a hash of the proof data, which will be used as the header of for the bbs signature operation.\\

The input for this algorithm is the proof config as canonical staments \textbf{canonical\_proof\_config}, a set of mandatory pointers \textbf{mandatory\_attributes} and a transformed document \textbf{transformed\_data}.\\

The output is an object \textbf{hash\_data} which is a copy of the \textbf{transformed\_data} with new key-value pairs.\\

We follow these steps:
\begin{itemize}
	\item Set \textbf{proof\_hash} as the result of the hashing of \textbf{canonical\_proof\_config} using SHA-256.
	\item Set the \textbf{mandatory\_hash} as the result of the hashing of \textbf{transformed\_document.mandatory} using SHA-256. Be aware that mandatory is an array that first needs to be concatenated. In this thesis we don't use any delimiter for concatenation
	\item Initialise \textbf{hash\_data} as a copy of \textbf{transformed\_data}, then adding the key-value pairs \textit{proofHash} - \textbf{proof\_hash} and \textit{mandatoryHash} - \textbf{mandatory\_hash}
	\item Return \textbf{hash\_data}
\end{itemize}

With this object we now have all data to be able to create a bbs signature.

\section{Generate and serialize the signature}
\label{subsubsec:sign}
This algorithm shows how the different data prepared in the algorithms before, is used to generate a valid bbs signature.\\

The inputs for this algorithm are the hashed proof data \textbf{hash\_data} and a set of mandatory pointers \textbf{mandatory\_attributes}.\\

We follow these steps:
\begin{itemize}
	\item Set \textbf{proof\_hash}, \textbf{mandatory\_hash}, \textbf{non\_mandatory} and \textbf{hmac\_key} to the corresponding value in \textbf{hash\_data}.
	\item Set \textbf{bbs\_header} to the concatenation of \textbf{proof\_hash} and \textbf{mandatory\_hash}
	\item Set \textbf{bbs\_messages} to the messages in the \textbf{non\_mandatory} array, which are represented as byte arrays encoded with UTF-8
	\item Set \textbf{bbs\_signature} to the result of the bbs signature operation, defined in chapter 3.5.1 of the BBS Siganture Scheme\cite{bbs-signature-scheme}.
	\item Initialise \textbf{proof\_value} to the bbs proof header bytes 0xd9, 0x5d and 0x02
	\item Set \textbf{components} to an array containing(in this order): \textbf{bbs\_signature}, \textbf{bbs\_header}, \textbf{public\_key}, \textbf{hmac\_key}, \textbf{mandatory\_pointers}
	\item Set \textbf{cbor} to result of the application of the CBOR-encoding\cite{cbor}, where CBOR tagging \textbf{Must NOT} be used.
	\item Set \textbf{proof\_value} to the concatenation of \textbf{proof\_value} and \textbf{cbor}
	\item Set \textbf{base\_proof} to the value of the base64-url encoding defined in chapter 5 of "RFC 4648"\cite{base64} but without padding as mentioned in chapter 3.2 of the same RFC.
	\item Return \textbf{base\_proof}
\end{itemize}

We now have a valid BBS signature, which we can append to the VC.

\section{Add signature to VC}
\label{subsubsec:signed}
This algorithm adds a bbs signature to a VC document.

The input to this algorithm is a VC \textbf{vc}, the base proof \textbf{base\_proof} and a proof config \textbf{proof\_config}.

The output of this algorithm is a signed VC \textbf{vc} with the proof information attached.

We follow these steps:
\begin{itemize}
	\item Add a new key-value pair \textit{proof} - \textbf{proof\_config} to \textbf{vc}
	\item Add a new key-value pair \textit{proofValue} - \textbf{base\_proof} to \textbf{vc.proof}
	\item Return \textbf{vc}
\end{itemize}

And with that we now have a fully functional VC which was signed with BBS.


\chapter{Derive selective disclosure VP}

\section{Derive VP}
\label{subsub:derivevp}

This algorithm handles all the data for a selective disclosure proof creation.\\

The inputs of this algorithm are a VC \textbf{vc}, the proof \textbf{base\_proof}, the selective pointers \textbf{selective\_pointers} and a presentation header \textbf{ph}.\\

The output is an altered VC \textbf{reavealed\_document} only containing the data to be revealed.\\

We follow these steps:
\begin{itemize}
	\item Set \textbf{bbs\_proof}, \textbf{label\_map}, \textbf{mandatory\_indexes}, \textbf{selective\_indexes} and \textbf{reavealed\_document} to their corresponding entry in the response object from the algorithm described in chapter \ref*{subsub:createdisclosuredata}, with the inputs \textbf{vc}, \textbf{base\_proof}, \textbf{selective\_pointers} and \textbf{ph}
	\item Set \textbf{new\_proof} to a copy of \textbf{base\_proof}
	\item Replace \textbf{new\_proof.proofValue} with the result of calling the algorithm in chapter \ref{subsub:parsebaseproof}, with the inputs \textbf{bbs\_proof}, \textbf{label\_map}, \textbf{mandatory\_indexes} and \textbf{selective\_indexes}
	\item Set \textbf{reavealed\_document.proof} to \textbf{new\_proof}
	\item Return \textbf{reavealed\_document}
\end{itemize}

\section{Create disclosure data}
\label{subsub:createdisclosuredata}

This algorithm creates the data for the selective disclosure of a VP.\\

The inputs of this algorithm are a VC \textbf{vc}, the proof \textbf{base\_proof}, the selective pointers \textbf{selective\_pointers} and a presentation header \textbf{ph}.\\

The output is an object containing the BBS proof, a label map, mandatory indexes, selective indexes and the revealed document.\\

We follow these steps:
\begin{itemize}
	\item Set \textbf{bbs\_signature}, \textbf{bbs\_header}, \textbf{public\_key}, \textbf{hmac\_key} and \textbf{mandatory\_pointers} to their respective value in the response of the algorithm described in chapter \ref{subsub:parsebaseproof}
	\item Define \textbf{label\_map\_factory\_function} as the function which is returned from the \textbf{create\_shuffled\_id\_label\_map\_function} defined in chapter \ref{sec:shuffledlabelmap} with the input \textbf{hmac\_key}
	\item Set \textbf{combined\_pointers} to the concatenation of \textbf{mandatory\_pointers} and \textbf{selective\_pointers}
	\item Set \textbf{group\_definitions} to an object with following key-value pairs: \textit{mandatory} - \textbf{mandatory\_pointers}, \textit{selective} - \textbf{selective\_pointers} and \textit{combined} - \textit{combined\_pointers}
	\item Set \textbf{groups} and \textbf{label\_map} to \textit{response.groups} and \textit{response.labelMap} respectively of the \textit{canonicalizeAndGroup} algorithm specified in the DataIntegrity for ECDSA specification\cite{ecdsa}, passing the \textbf{label\_map\_factory\_function}, \textbf{group\_definitions} and \textbf{vc}.
	\item Set \textbf{combined\_match} to \textbf{groups.combined.matching}
	\item Set \textbf{mandatory\_match} to \textbf{groups.mandatory.matching}
	\item Set \textbf{combined\_indexes} to the ordered list of \textbf{combined\_match.keys}
	\item Set \textbf{mandatory\_indexes} to an empty array. We want to compute the position of the mandatory indexes in the \textbf{combined\_match} array, so that the verifier knows which indexes were mandatory to reveal
	\item For each key in \textbf{mandatory\_match} find its index in \textbf{combined\_indexes} and add it to \textbf{mandatory\_indexes}.
	\item Set \textbf{selective\_match} to \textbf{groups.selective.matching}
	\item Set \textbf{mandatory\_non\_match} to \textbf{groups.mandatory.nonMatching}
	\item Set \textbf{non\_mandatory\_indexes} to the ordered list of \textbf{mandatory\_non\_match.keys}
	\item Set \textbf{selective\_indexes} to an empty array. This time we want to compute the position of the selective indexes in the \textbf{non\_mandatory\_indexes} list. This list will be used for the bbs proof generation process, to define which messages are going to be revealed.
	\item For each key in \textbf{selective\_match} find its index in \textbf{non\_mandatory\_indexes} and add it to \textbf{selective\_indexes}
	\item Set \textbf{bbs\_messages} to the values of \textbf{non\_mandatory.values} as byte arrays
	\item Set \textbf{bbs\_proof} to the result of the bbs proof operation defined in chapter 3.5.3 of \textit{The BBS Signature Scheme}\cite{bbs-signature-scheme}, with the input \textbf{public\_key}, \textbf{bbs\_signature}, \textbf{bbs\_header}, \textbf{ph}, \textbf{bbs\_messages} and \textbf{selective\_indexes}
	\item Set \textbf{reavealed\_document} to the result of the algorithm described in chapter 3.4.13 of \textit{Data Integrity ECDSA Cryptosuites v1.0}\cite{ecdsa}, with the inputs \textbf{vc} and \textbf{combined\_pointers}
	\item Set \textbf{deskolemized\_n\_quads} to the joined string from the \textbf{groups.combined.deskolemizedNQuads} array. In this thesis we do not use a delimiter
	\item Set \textbf{canonical\_id\_map} to the result of the JSON-LD canonicalization algorithm
	\item Set \textbf{verifier\_label\_map} to an empty map. This maps the canonical blank node identifiers like \textit{c14n0} from the revealed vc to the blank node identifiers created by the issuer
	\item For each entry in \textbf{canonical\_id\_map}:
	\begin{itemize}
		\item Add an entry to \textbf{verifier\_label\_map} where the key is the current value of the \textbf{canonical\_id\_map} entry and the value is the entry where the key the in \textbf{label\_map} is the same as the current key from \textbf{canonical\_id\_map}
	\end{itemize}
	\item Return an object with following key-value pairs: \textit{bbsProof} - \textbf{bbs\_proof}, \textit{verifierLabelMap} - \textbf{label\_map}, \textit{mandatoryIndexes} - \textbf{mandatory\_indexes}, \textit{selectiveIndexes} - \textbf{selective\_indexes} and \textit{revealedDocument} - \textbf{reavealed\_document}
\end{itemize}

\section{Parse the base proof}
\label{subsub:parsebaseproof}

This algorithm parses a base proof value back to objects.\\

The input of this algorithm is a base proof \textbf{base\_proof}.\\

The output is an object \textbf{parsed\_base\_proof} containing the parsed base proof values.\\

We follow these steps to parse the base proof:
\begin{itemize}
	\item Check that the \textbf{base\_proof} start with a \textit{u}, indicating that it is a base64-url encoded string.
	\item Set \textbf{decoded\_proof\_value} to the result of the base64-url decoding with no padding as described in \textit{The Base16, Base32, and Base64 Data Encodings}\cite{base64}
	\item Check that \textbf{decoded\_proof\_value} starts with the proof header bytes \textit{0xd9}, \textit{0x5d} and \textit{0x02}
	\item Set \textbf{components} to the cbor decoding described in \textit{RFC 8949}\cite{cbor}, starting with the fourth byte in \textbf{decoded\_proof\_value}
	\item Set \textbf{base\_proof\_object} to the key-value pairs \textit{bbs\_signature} - \textbf{components[0]}, \textit{bbs\_header} - \textbf{components[1]}, \textit{public\_key} - \textbf{components[2]}, \textit{hmac\_key} - \textbf{components[3]} and \textit{mandatory\_pointers} - \textbf{components[4]}
	\item Return \textbf{base\_proof\_object}
\end{itemize}

With that we parsed the base proof bytes back to objects.

\section{Serialize the proof value}
\label{subsub:serializebaseproof}

This algorithm serializes a proof value.\\

The inputs of this algorithm are a BBS proof \textbf{bbs\_proof}, a label map \textbf{label\_map}, an array of mandatory indexes \textbf{mandatory\_indexes}, an array of selective indexes \textbf{selective\_pointers} and a presentation header \textbf{ph}.\\

The result of this algorithm is the serialized proof value \textbf{proof\_value}.\\

We follow these steps:
\begin{itemize}
	\item Set \textbf{compressed\_label\_map} to the result of calling the algorithm in chapter 3.5.5 of the \textit{Data Integrity ECDSA Cryptosuites v1.0}\cite{ecdsa} with \textbf{label\_map} as the input
	\item Initialize \textbf{proof\_value} as an byte array, where the first three bytes are \textit{0xd9}, \textit{0x5d} and \textit{0x03}
	\item Set \textbf{components} to an array containing in this order: \textbf{bbs\_proof}, \textbf{compressed\_label\_map}, \textbf{mandatory\_indexes}, \textbf{selective\_pointers} and \textbf{ph}
	\item Set \textbf{cbor} to result of the application of the CBOR-encoding\cite{cbor} on the components array, where CBOR tagging \textbf{Must NOT} be used.
	\item Concat \textbf{proof\_value} and \textbf{cbor}
	\item Set \textbf{derived\_proof} to the value of the base64-url encoding defined in chapter 5 of "RFC 4648"\cite{base64} but without padding as mentioned in chapter 3.2 of the same RFC
	\item Return \textbf{proof\_value}
\end{itemize}


\chapter{Verify selective disclosure VP}

\section{Create Verify data}
\label{subsub:createverifydata}
This algorithm generates the data for the proof verification process.\\

The input of this algorithm are the VP without the proof value \textbf{document}, the proof \textbf{proof}, and the proof without \textit{proofValue} \textbf{proof\_config}.\\

The output of this algorithm is a label map containing the BBS proof, the proof hash, the non mandatory indexes, the mandatory hash, the selective indexes and the feature options.\\

We follow these steps:
\begin{itemize}
	\item Set \textbf{proof\_hash} to the hashed (using SHA-256) canonical representation of the \textbf{proof\_config} using the \textit{Universal RDF Dataset Canonicalization Algorithm}\cite{rdf}
	\item Set \textbf{bbs\_proof}, \textbf{label\_map}, \textbf{mandatory\_indexes}, \textbf{selective\_indexes}, \textbf{} and \textbf{feature\_option} to their respective entries in the result of the algorithm described in chapter \ref{subsub:parseproof} with \textbf{proof} as the input
	\item Set \textbf{label\_map\_factory\_function} to the algorithm in chapter 3.4.3 of \textit{Data Integrity ECDSA Cryptosuites v1.0}\cite{ecdsa}
	\item Set \textbf{nquads} to the result of the algorithm \textit{labelReplacementCanonicalize} described in chapter 3.4.1 of \textit{Data Integrity ECDSA Cryptosuites v1.0}\cite{ecdsa} with the inputs \textbf{document} and \textbf{label\_map\_factory\_function}
	\item Set \textbf{mandatory} and \textbf{non\_mandatory} to an empty array
	\item For each entry in nquads, the key is \textbf{key} and the value is \textbf{value}, separate them into mandatory and non mandatory
	\begin{itemize}
		\item If \textbf{mandatory\_indexes} contains \textbf{key}, add \textbf{value} to \textbf{mandatory}
		\item Else add \textbf{value} to \textbf{non\_mandatory}
	\end{itemize}
	\item Set \textbf{mandatory\_hash} to hash of \textbf{mandatory} using SHA-256. We concatenate the elements of the array without a separator
	\item Return a map containing the key-value pairs: \textit{bbsProof} - \textbf{bbs\_proof}, \textit{proofHash} - \textbf{proof\_hash}, \textit{nonMandatory} - \textbf{non\_mandatory}, \textit{mandatoryHash} - \textbf{mandatory\_hash}, \textit{selectiveIndexes} - \textbf{selective\_indexes}, \textit{presentationHeader} - \textbf{ph} and \textit{featureOption} - \textbf{feature\_option}
\end{itemize}

\section{Parse Proof Value}
\label{subsub:parseproof}
This algorithm parses the proof value.\\

The input is a proof value \textbf{proof\_value}.\\

The output is a map containing the BBS proof, a label map, the mandatory indexes, the selective indexes, the presentation header and the feature options\\

We follow these steps:
\begin{itemize}
	\item Make sure that \textbf{proof\_value} starts with \textit{u}, which indicates that it is a base64-url encoded value with no padding
	\item Set \textbf{decoded\_proof\_value} to the decoding of \textbf{proof\_value}
	\item Make sure that the three first bytes of \textbf{decoded\_proof\_value} are \textit{0xd9}, \textit{0x5d} and \textit{0x03} in that order
	\item Set \textbf{feature\_option} to \textit{baseline}
	\item Set \textbf{components} to the result of the CBOR decoding of the \textbf{decoded\_proof\_value} starting with the fourth byte. The resulting array must have 5 elements
	\item Replace the second object in \textbf{components} with the result of the algorithm in chapter 3.5.6 of \textit{Data Integrity ECDSA Cryptosuites v1.0}\cite{ecdsa}, with the second object of \textbf{components} as input value
	\item Return a map containing the key-value pairs: \textit{bbsProof} - \textbf{components[0]}, \textit{labelMap} - \textbf{components[1]}, \textit{mandatoryIndexes} - \textbf{components[2]}, \textit{selectiveIndexes} - \textbf{components[3]}, \textit{presentationHeader} - \textbf{components[5]} and \textit{featureOption} - \textbf{feature\_option}
\end{itemize}


\appendix



\chapter{First appendix Chapter}

\bibliographystyle{plain}
\bibliography{refs}

\end{document}
