%============================ Project Managenent Document ================================
% define document class
\documentclass[
	a4paper               % paper format
%	,10.5pt               % fontsize
%	,BCOR=18mm            % Binding correction
	,bibliography=totoc   % If enabled add bibliography to TOC
	,listof=totoc         % If enabled add lists to TOC
%	,bilingual
	,monolingual
]{bfhthesis}              % KOMA-script report

\usepackage[
	hidelinks,
	pdfusetitle,
]{hyperref}

\begin{document}

\frontmatter

\title{Bachelor's Thesis}
\subtitle{Extended BBS?: Management Document}
\author{Joël Gabriel Robles Gasser}
\institution{Bern University of Applied Sciences}
\department{Engineering and Computer Science}
\institute{Computer Science}
\version{0.1}
\advisor{Prof. Dr. Annett Laube \and Prof. Dr. Reto Koenig}
\expert{Dr. Andreas Spichiger}
\degreeprogram{Bachelor of Science in Computer Science}

%----------------  BFH tile page   -----------------------------------------
\maketitle

\addchap{Abstract}
Here an abstract might be placed.


%------------ TABLEOFCONTENTS ----------------
\tableofcontents

%------------ START MAIN PART ----------------
\mainmatter

\chapter{Introduction}

\chapter{Goals}

\chapter{Recap of BLS12-381, BBS and Pairings}
This chapter contains a quick recap about BBS and BLS12-381.
The Mathematics of these topics are out of scope for this Thesis, so only the top level ideas will be discussed.

\section{BLS12-381}
\begin{figure}[h]
    \centering
	\includegraphics[width=4cm]{example-image-duck}
	\caption{The BLS12-381 curve}
	\label{fig:bls12381}
\end{figure}
The Barreto-Lynn-Scott Curves \cite{pairing-friendly-curves} are a group of pairing friendly curves. 
Specifically the BLS12-381 Curve is used in the BBS Signature Scheme \cite{bbs-signature-scheme}.
The 12 in the name comes from the embedding degree, the 381 is the amount of bits necessary to represent a point on the curve.
This Curve is defined as with the following equation $y^2 = x^3 + 4$, where x and y are coordinates on the Field $F_p$.

\subsection{Field Extensions}
$G_1$ is the largest is the largest prime order subgroup of the BLS curve.
For the Pairings discussed in section \ref{sec:pairing} we need multiple points on curves as inputs.
Besides a point on $G_1$ we also need a point on $G_2$.
This $G_2$ is an Extension of the Field $F_p$ into the Field $F_{p^2}$.
This alters the curve equation a bit to $y^2 = x^3 + 4(u + 1)$ where x and y are no longer coordinates but are polynoms of the second order.
Both $G_1$ and $G_2$ are additive Groups.
For the Pairings we also need a third Group, this time in the Field $F_{p^{12}}$.
This is Group is a multiplicative group called $G_T$ in section \ref{sec:pairing}, where x and y are polynoms of the $12^{th}$ order.
With this we now know all the necessary groups for \ref{sec:bbs}.
The Basepoints (generators) of both $G_1$ and $G_2$ are defined in \cite{pairing-friendly-curves} as follows (Big-endian order encoded as HEX):\newline

\boldmath$G_1$(BP1):\newline
x = 0x17f1d3a73197d7942695638c4fa9ac0fc3688c4f9774b905a14e3a3f171bac586c55e83ff97a1aeffb3af00adb22c6bb
y = 0x08b3f481e3aaa0f1a09e30ed741d8ae4fcf5e095d5d00af600db18cb2c04b3edd03cc744a2888ae40caa232946c5e7e1\newline\newline
\boldmath$G_2$(BP2):\newline
$x_O$ = 0x024aa2b2f08f0a91260805272dc51051c6e47ad4fa403b02b4510b647ae3d1770bac0326a805bbefd48056c8c121bdb8
$x_1$ = 0x13e02b6052719f607dacd3a088274f65596bd0d09920b61ab5da61bbdc7f5049334cf11213945d57e5ac7d055d042b7e
$y_O$ = 0x0ce5d527727d6e118cc9cdc6da2e351aadfd9baa8cbdd3a76d429a695160d12c923ac9cc3baca289e193548608b82801
$y_1$ = 0x0606c4a02ea734cc32acd2b02bc28b99cb3e287e85a763af267492ab572e99ab3f370d275cec1da1aaa9075ff05f79be

\section{BBS}
\label{sec:bbs}
\begin{figure}[h]
    \centering
	\includegraphics[width=4cm]{example-image-duck}
	\caption{The Actors of BBS and their connection}
	\label{fig:bbstriangle}
\end{figure}
The BBS Signature Scheme \cite{bbs-signature-scheme} is a multi-message signature scheme with support for selective disclosure and proof of knowledge of the signature, thus proving unlinkability between different verfiers. 
Figure \ref{fig:bbstriangle} shows the flows between the different actors.
For this thesis we define following names for the actors:
\begin{itemize}
	\item Issuer - Issues a signature on a set of messages
	\item Holder - Holds the set of messages as well as the signature. Also generates the Proof for the verifier.
	\item Verifier - Gets the disclosed messages as well as the proof, which he then verifies
\end{itemize}
For the key generation a random scalar \textbf{SK} and the Base Point of $G_2$ are needed.
Thus the public key (calculated as $SK*BP2$) results in Point on $G_2$.
This allows for all other calculations (like the signature or the proof) to be done in $G_1$ which in turn makes the algorithm more efficient.


\section{Pairings}
\label{sec:pairing}
For the verification of the BBS signature and proof Pairing-functions are used. The most important part of these functions are their billiniarity i.e.,\newline
\begin{equation}
		e(A, B + B') = e(A, B)e(A, B') \text{and} e(a * A, b * B) = e(A, B)^{ab}
\end{equation}

With these characteristics we can understand the equations for verifying the signature and proof.\newline

\textbf{Example 1. BBS Signature verification}
\begin{equation}
	\begin{split}
		Identity_{GT} & = e(A, W + BP2 * e) * e(B, -BP2) \\
		& = e(A, BP2 * e)e(A, W)e(B, BP2)^{-1} \\
		& = e(A, BP2)^ee(A, SK * BP2)e(B, BP2)^{-1} \\
		& = e(A, BP2)^ee(A, BP2)^{SK}e(B, BP2)^{-1} \\
		& = e(A, BP2)^{e + SK}e(B, BP2)^{-1} \\
		& = e(B * \frac{1}{sk * e}, BP2)^{e + SK}e(B, BP2)^{-1} \\
		& = e(B, BP2)^{\frac{e + SK}{e + SK}}e(B, BP2)^{-1} \\
		& = e(B, BP2)^1e(B, BP2)^{-1} \\
		& = e(B, BP2)^0 \\
		& = 1
	\end{split}
\end{equation}

\chapter{BBS with VC's}

\section{Use Case}

\appendix

\chapter{First appendix Chapter}



\bibliographystyle{plain}
\bibliography{refs}

\end{document}
