%============================ Project Managenent Document ================================
% define document class
\documentclass[
	a4paper               % paper format
%	,10.5pt               % fontsize
%	,BCOR=18mm            % Binding correction
	,bibliography=totoc   % If enabled add bibliography to TOC
	,listof=totoc         % If enabled add lists to TOC
%	,bilingual
	,monolingual
	twoside=false,
]{bfhthesis}              % KOMA-script report

\setcounter{secnumdepth}{4}

\PassOptionsToPackage{hyphens}{url}\usepackage[
	hidelinks,
	pdfusetitle,
]{hyperref}
\usepackage{tikzducks}
\usepackage{amsmath}
\usepackage{listings}

\LoadBFHModule{boxes}

\colorlet{punct}{red!60!black}
\definecolor{background}{HTML}{EEEEEE}
\definecolor{delim}{RGB}{20,105,176}
\colorlet{numb}{magenta!60!black}

\lstdefinelanguage{json}{
    basicstyle=\normalfont\ttfamily,
    numbers=left,
    numberstyle=\scriptsize,
    stepnumber=1,
    numbersep=8pt,
    showstringspaces=false,
    breaklines=true,
    frame=lines,
	postbreak=\mbox{\textcolor{red}{$\hookrightarrow$}\space},
    backgroundcolor=\color{background},
    literate=
      {:}{{{\color{punct}{:}}}}{1}
      {,}{{{\color{punct}{,}}}}{1}
      {\{}{{{\color{delim}{\{}}}}{1}
      {\}}{{{\color{delim}{\}}}}}{1}
      {[}{{{\color{delim}{[}}}}{1}
      {]}{{{\color{delim}{]}}}}{1},
}

\lstdefinelanguage{canon}{
    basicstyle=\normalfont\ttfamily,
    numbers=left,
    numberstyle=\scriptsize,
    stepnumber=1,
    numbersep=8pt,
    showstringspaces=false,
    breaklines=true,
    frame=lines,
    backgroundcolor=\color{background},
	postbreak=\mbox{\textcolor{red}{$\hookrightarrow$}\space},
}

\hyphenation{ve-ri-fi-ca-ti-on}

\begin{document}

\frontmatter

\title{Bachelor's Thesis}
\subtitle{Unlinkability of Verfiable Credentials in a practical approach
: Project Management Document}
\author{Joël Gabriel Robles Gasser}
\institution{Bern University of Applied Sciences}
\department{Engineering and Computer Science}
\institute{Computer Science}
\version{0.1}
\advisor{Prof. Dr. Annett Laube \and Prof. Dr. Reto Koenig}
\expert{Dr. Andreas Spichiger}
\degreeprogram{Bachelor of Science in Computer Science}

%----------------  BFH tile page   -----------------------------------------
\maketitle

\addchap{Abstract}
Here an abstract might be placed.


%------------ TABLEOFCONTENTS ----------------
\tableofcontents

%------------ START MAIN PART ----------------
\mainmatter

\chapter{Goals}


\chapter{Risks}
In this chapter we define Risks that may happen in this thesis.

\section{Project Risks}

\subsection{VC}
\begin{itemize}
	\item Risk: There is the possibility that the structure of VCs breaks the unlinkability of BBS
	\item Solution: If that is the case, the stucture of VCs needs to be reworked so there is no linkability
	\item Possibility: Low
\end{itemize}

\subsection{OIDC4VP}
\begin{itemize}
	\item Risk: There is the possibility that the implementation of OIDC4VP leaks data
	\item Solution: If that is the case, the structure of the OIDC4VC protocol needs to be reworked so there are no more data leaks
	\item Possibility: Low
\end{itemize}

\subsection{OIDC4VP}
\begin{itemize}
	\item Risk: There is the possibility that the implementation of OIDC4VP leaks data
	\item Solution: If that is the case, the structure of the OIDC4VC protocol needs to be reworked so there are no more data leaks
	\item Possibility: Low
\end{itemize}
\begin{itemize}
	\item Risk: There is the possibility that the implementation of OIDC4VP breaks the unlinkability of BBS
	\item Solution: If that is the case, the structure of the OIDC4VP protocol needs to be reworked so there is no linkability
	\item Possibility: Medium
\end{itemize}

\subsection{Psuedonyms}
\begin{itemize}
	\item Risk: There is the possibility that the implementation of Pseudonyms breaks the unlinkability of BBS
	\item Solution: If that is the case, the structure of Pseudonyms needs to be reworked, so there is no linkability
	\item Possibility: Low
\end{itemize}

\subsection{Link Secrets}
\begin{itemize}
	\item Risk: There is the possibility that the implementation of Link Secrets breaks the unlinkability of BBS
	\item Solution: If that is the case, the implementation of the Link Secrets needs to be reworked, so there is no linkability
	\item Possibility: Medium
\end{itemize}
\begin{itemize}
	\item Risk: There is the possibility that the implementation of Link Secrets allows the holder of multiple VCs to Link the together on a different Secret
	\item Solution: If that is the case, the implementation of the Link Secrets needs to be reworked so that this is no longer possible
	\item Possibility: Low
\end{itemize}

\section{Environmental risks}

\subsection{Sickness}
\begin{itemize}
	\item Risk: There is a possibility that I may become sick
	\item Solution: If the sickness is less than 1 Week, there is a Buffer in the project plan at the end of the semester for that. If it's more than 1 Week, there is a chance that the Project would need to be moved to another semester
	\item Possibility: Medium
\end{itemize}

\subsection{Hardware}
\begin{itemize}
	\item Risk: There is a possibility that the Hardware used (Laptop) may break due to unknown reasons
	\item Solution: The Deliverables are backed u to GitHub and mirrored to the BFH-TI GitLab. There is also a Backup on other Hardware. There also Backup Hardware if the main Hardware would break
	\item Possibility: Low
\end{itemize}
\begin{itemize}
	\item Risk: There is the possibility that the Backups are not accessible
	\item Solution: For that case there is an also a Backup on different Devices
	\item Possibility: Low
\end{itemize}

\subsection{Project Plan/Ideas}
\begin{itemize}
	\item Risk: There is the possibility that the project plan is bad, so that the Planned time frames are to short
	\item Solution: In that case there is 1 week buffer at the end of the semester. If that is not enough time, then there needs to be an explanation in the documentation why that happened
	\item Possibility: Medium
\end{itemize}
\begin{itemize}
	\item Risk: There is the possibility that the expert of the project may bring new Ideas into the project
	\item Solution: If those ideas further the projects goals, they may be added to the project plan. In that case the Project plan needs to be reworked
	\item Possibility: Low
\end{itemize}


\chapter{Found Issues}
% \chapter{Recap of BLS12-381, BBS and Pairings}
% This chapter contains a quick recap about BBS and BLS12-381.
% The Mathematics of these topics are out of scope for this Thesis, so only the top level ideas will be discussed.

% \section{BLS12-381}
% \begin{figure}[h]
%     \centering
% 	\includegraphics[width=4cm]{example-image-duck}
% 	\caption{The BLS12-381 curve}
% 	\label{fig:bls12381}
% \end{figure}
% The Barreto-Lynn-Scott Curves \cite{pairing-friendly-curves} are a group of pairing friendly curves. 
% Specifically the BLS12-381 Curve is used in the BBS Signature Scheme \cite{bbs-signature-scheme}.
% The 12 in the name comes from the embedding degree, the 381 is the amount of bits necessary to represent a point on the curve.
% This Curve is defined as with the following equation $y^2 = x^3 + 4$, where x and y are coordinates on the Field $F_p$.

% \subsection{Field Extensions}
% $G_1$ is the largest is the largest prime order subgroup of the BLS curve.
% For the Pairings discussed in section \ref{sec:pairing} we need multiple points on curves as inputs.
% Besides a point on $G_1$ we also need a point on $G_2$.
% This $G_2$ is an Extension of the Field $F_p$ into the Field $F_{p^2}$.
% This alters the curve equation a bit to $y^2 = x^3 + 4(u + 1)$ where x and y are no longer coordinates but are polynoms of the second order.
% Both $G_1$ and $G_2$ are additive Groups.
% For the Pairings we also need a third Group, this time in the Field $F_{p^{12}}$.
% This is Group is a multiplicative group called $G_T$ in section \ref{sec:pairing}, where x and y are polynoms of the $12^{th}$ order.
% With this we now know all the necessary groups for \ref{sec:bbs}.
% The Basepoints (generators) of both $G_1$ and $G_2$ are defined in \cite{pairing-friendly-curves} as follows (Big-endian order encoded as HEX):\newline

% \boldmath$G_1$(BP1):\newline
% x = 0x17f1d3a73197d7942695638c4fa9ac0fc3688c4f9774b905a14e3a3f171bac586c55e83ff97a1aeffb3af00adb22c6bb
% y = 0x08b3f481e3aaa0f1a09e30ed741d8ae4fcf5e095d5d00af600db18cb2c04b3edd03cc744a2888ae40caa232946c5e7e1\newline\newline
% \boldmath$G_2$(BP2):\newline
% $x_O$ = 0x024aa2b2f08f0a91260805272dc51051c6e47ad4fa403b02b4510b647ae3d1770bac0326a805bbefd48056c8c121bdb8
% $x_1$ = 0x13e02b6052719f607dacd3a088274f65596bd0d09920b61ab5da61bbdc7f5049334cf11213945d57e5ac7d055d042b7e
% $y_O$ = 0x0ce5d527727d6e118cc9cdc6da2e351aadfd9baa8cbdd3a76d429a695160d12c923ac9cc3baca289e193548608b82801
% $y_1$ = 0x0606c4a02ea734cc32acd2b02bc28b99cb3e287e85a763af267492ab572e99ab3f370d275cec1da1aaa9075ff05f79be

% \section{BBS}
% \label{sec:bbs}
% \begin{figure}[h]
%     \centering
% 	\includegraphics[width=4cm]{example-image-duck}
% 	\caption{The Actors of BBS and their connection}
% 	\label{fig:bbstriangle}
% \end{figure}
% The BBS Signature Scheme \cite{bbs-signature-scheme} is a multi-message signature scheme with support for selective disclosure and proof of knowledge of the signature, thus proving unlinkability between different verfiers. 
% Figure \ref{fig:bbstriangle} shows the flows between the different actors.
% For this thesis we define following names for the actors:
% \begin{itemize}
% 	\item Issuer - Issues a signature on a set of messages
% 	\item Holder - Holds the set of messages as well as the signature. Also generates the Proof for the verifier.
% 	\item Verifier - Gets the disclosed messages as well as the proof, which he then verifies
% \end{itemize}
% For the key generation a random scalar \textbf{SK} and the Base Point of $G_2$ are needed.
% Thus the public key (calculated as $SK*BP2$) results in Point on $G_2$.
% This allows for all other calculations (like the signature or the proof) to be done in $G_1$ which in turn makes the algorithm more efficient.


% \section{Pairings}
% \label{sec:pairing}
% For the verification of the BBS signature and proof Pairing-functions are used. The most important part of these functions are their billiniarity i.e.,\newline
% \begin{equation}
% 		e(A, B + B') = e(A, B)e(A, B') \text{and} e(a * A, b * B) = e(A, B)^{ab}
% \end{equation}

% With these characteristics we can understand the equations for verifying the signature and proof.\newline

% \textbf{Example 1. BBS Signature verification}
% \begin{equation}
% 	\begin{split}
% 		Identity_{GT} & = e(A, W + BP2 * e) * e(B, -BP2) \\
% 		& = e(A, BP2 * e)e(A, W)e(B, BP2)^{-1} \\
% 		& = e(A, BP2)^ee(A, SK * BP2)e(B, BP2)^{-1} \\
% 		& = e(A, BP2)^ee(A, BP2)^{SK}e(B, BP2)^{-1} \\
% 		& = e(A, BP2)^{e + SK}e(B, BP2)^{-1} \\
% 		& = e(B * \frac{1}{sk * e}, BP2)^{e + SK}e(B, BP2)^{-1} \\
% 		& = e(B, BP2)^{\frac{e + SK}{e + SK}}e(B, BP2)^{-1} \\
% 		& = e(B, BP2)^1e(B, BP2)^{-1} \\
% 		& = e(B, BP2)^0 \\
% 		& = 1
% 	\end{split}
% \end{equation}

\chapter{First appendix Chapter}



\bibliographystyle{plain}
\bibliography{refs}

\end{document}
