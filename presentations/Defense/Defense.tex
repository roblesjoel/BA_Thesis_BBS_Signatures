\documentclass[
	german,%globale Übergabe der Hauptsprache
%	logofile=example-image-duck, %Falls die Logo Dateien nicht vorliegen
	authorontitle=true,
	]{bfhbeamer}


\usepackage[main=german]{babel}
\usepackage{caption}

% Der folgende Block ist nur bei pdfTeX auf Versionen vor April 2018 notwendig
\usepackage{iftex}
\ifPDFTeX
\usepackage[utf8]{inputenc}%kompatibilität mit TeX Versionen vor April 2018
\fi


%Makros für Formatierungen der Doku
%Im Allgemeinen nicht notwendig!
\let\code\texttt

\title{Bachelor Thesis}
\subtitle{Unlinkability of Verifiable Credentials in a practical approach}
\author[J. Robles]{Joel Robles}
\institute{TI}
\titlegraphic*{\includegraphics{example-image-duck}}%is only used with BFH-graphic and BFH-fullgraphic
\date{June 5, 2024}

% \setbeamertemplate{page number in foot}[framenumber]

%Activate the output of a frame number:
\setbeamertemplate{footline}[frame number]


\AtBeginSection{\sectionpage}

\begin{document}

\maketitle

\begin{frame}{Inhaltsverzeichnis}
    \tableofcontents
\end{frame}

\section{Ziel}

\begin{frame}{Was ist das Ziel?}
    \centering
    Die Analyse, ob eine Implementation von Verifiable Credentials mit dem BBS Signature Scheme in der realen Welt, unverknüpfbarkeit beibehält
\end{frame}

\section{Self-sovereign Identity}

\begin{frame}{Self-sovereign Identity (SSI)}
    \begin{itemize}
        \item Ist ein Konzept wo eine Person (\textbf{holder}) entscheiden kann, wer was über sie wissen darf
        \item Holders dürfen wählen was sie offenbaren und was nicht, auch bekannt als \textbf{selective disclosure}
        \item Erstes Problem:
        \begin{itemize}
            \item Holder zeigt eine Staatliche ID
            \item Ist eine Menge von Daten oder eine Menge von \textbf{Attributen}
            \item Die Person welche verifiziert sieht alle Attribute
        \end{itemize}
        \item Zweites Problem:
        \begin{itemize}
            \item Holder zeigt Attribute einer Person die diese verifizieren will, bekannt als \textbf{verifier}
            \item Holder zeigt die gleichen Attribute einem zweiten \textbf{verifier}
            \item Der holder kann ge-\textbf{linked} werden
        \end{itemize}
        \item Heutiger stand - Holder haben keine Kontrolle über ihre Attribute
        \item Zukünftiger stand dank SSI - Holder haben volle Kontrolle über ihre Attribute
    \end{itemize}
\end{frame}

\begin{frame}{Trust Triangle}
    \begin{columns}[onlytextwidth,T]
        \column{70mm}  
        \begin{itemize}
            \item Wie weis ein verifier das eine Menge von Attributen (\textbf{credential}) valid ist?
            \item Er vertraut dem issuer!
            \item Beispiel: Schweizer ID hat Hologramme
        \end{itemize}

        \column{70mm}

        \begin{figure}
            \centering
            \includegraphics[width=70mm]{../img/trusttriangle.png}
            \caption{Trust triangle}
        \end{figure}
        
    \end{columns}
\end{frame}

\section{Verifiable Credentials}

\begin{frame}{Verifiable Credentials (VC)}
    \begin{columns}[onlytextwidth,T]
        \column{70mm}  

    \begin{itemize}
        \item Verifiable Credentials sind eine digitale repräsentation von Physischen credentials
        \item JSON-LD repräsentiert Attribute als \textbf{key-value pairs}
        \item Beispiel:
        \begin{itemize}
            \item Vorname auf einer ID
            \item Repräsentiert als \{"first\_name": "John"\}
            \item ``first\_name'' ist der key und ``John'' ist der value
        \end{itemize}
    \end{itemize}

    \column{70mm}
    \begin{figure}
        \centering
        \includegraphics[width=70mm]{../img/VCexp.png}
        \caption{Beispiel VC}
    \end{figure}

    \end{columns}
\end{frame}

\begin{frame}{VCs and BBS}
    \begin{itemize}
        \item Warum werden sie \textbf{Verifiable} Credentials genannt?
        \item Der verifier kann ein VC, welches ihm präsentiert wurde (\textbf{Verifiable Presentation}), verifizieren, aufgrund Kryptographischen Signaturen
        \item Diese zeigen, dass das credential seit der Ausstellung nicht verändert wurde
        \item Wir nutzen das BBS Signature Scheme (\textbf{BBS}) 
        \item Diese Schema bietet \textbf{selective disclosure} und \textbf{unlinkability}
        \item Aber wie unlinkability? - Der Verifier braucht die Signatur
        \item BBS kann \textbf{proofs} generieren
        \item Diese beweisen das der holder die Signatur kennt, ohne diese zu offenbaren
        \item Weiter sind proofs unverknüpfbar zwischen jeder Generierung
    \end{itemize}
\end{frame}

\section{Verifiable Presentations}

\begin{frame}{Verifiable Presentation (VP)}
    \begin{columns}[onlytextwidth,T]
        \column{70mm}  
        \begin{itemize}
            \item Ein holder würde gerne ein VC präsentieren
            \item Dafür werden \textbf{Verifiable Presentations} genutzt
            \item BBS kann nur staments signieren
            \item Der \textbf{RDF} canonicalization Algorithmus, welcher statements aus key-value pairs generiert
        \end{itemize}

        \column{70mm}

        \begin{figure}
            \centering
            \includegraphics[width=70mm]{../img/VPcanon.png}
            \caption{Beispiel canonicalized VP}
        \end{figure}

    \end{columns}
\end{frame}

\begin{frame}{Der RDF Algorithmus}
    \centering
    JSON zu staments
    \begin{columns}[onlytextwidth,T]
        \column{70mm}  
        \begin{figure}
            \centering
            \includegraphics[width=70mm]{../img/VCexp.png}
            \caption{Beispiel VC}
        \end{figure}

        \column{70mm}  
        \begin{figure}
            \centering
            \includegraphics[width=70mm]{../img/statements.png}
            \caption{Beispiel statemetns}
        \end{figure}
    \end{columns}
\end{frame}

\begin{frame}{Der RDF Algorithmus}
    \centering
    Statements sortieren
    \begin{columns}[onlytextwidth,T]
        \column{70mm}  
        \begin{figure}
            \centering
            \includegraphics[width=70mm]{../img/statements.png}
            \caption{Beispiel statemetns}
        \end{figure}

        \column{70mm}  
        \begin{figure}
            \centering
            \includegraphics[width=70mm]{../img/hashed.png}
            \caption{Beispiel sortierung}
        \end{figure}
    \end{columns}
\end{frame}

\section{Sicherheitsüberlegungen von VC/VPs}

\begin{frame}{Permutation von statements}
    \begin{itemize}
        \item Holder präsentiert ein VP mit verborgenen Zivilstands-Attributen
        \item Holder heiratet bekommt ein neues VP mit geändertem Zivilistand
        \item Holder präsentiert das aktualisierte VP mit verborgenem Zivilstand
        \item \textbf{Datenleck}: Der verifier kann herausfinden, dass sich der Zivilistand geändert hat
        \item Damit das nicht passieren kann, muss der issuer immer die staments zufällig permutieren
        \item Der issuer muss die Permutation dem holder bekannt geben, aber \textbf{nie} dem verifier
    \end{itemize}
\end{frame}

\begin{frame}{Verknüpfbarkeit von Identifikatoren \& Metadaten}
    \begin{itemize}
        \item VCs können Metadaten wie Ablaufdatum beinhalten
        \item Falls das Ablaufdatum sehr genau ist (z.B. auf die Sekunde), führt dies zu Verknüpfbarkeit
        \item Ablaufdatum auf einen Tag genau, um die Verknüpfbarkeit zu umgehen
        \item VCs und VPs können auch ids für z.B. Entziehung beinhalten, welche zu Verknüpfbarkeit führen
        \item Dafür kan man Zero-Knowledge proofs verwenden, um zu zeigen das man nicht Teil einer Entzugs-Liste ist
    \end{itemize}
\end{frame}

\section{OpenID Connect for Verifiable Presentations}

\begin{frame}{Transport zwischen holder und verifier}
    \begin{figure}
        \centering
        \includegraphics[width=70mm]{../img/OIDC4VP.png}
        \caption{OpenID connect for Verifiable Presentations}
    \end{figure}
\end{frame}

\begin{frame}{OIDC4VP Fluss}
    \begin{figure}
        \centering
        \includegraphics[width=70mm]{../img/OIDC4VPFlow.png}
        \caption{OpenID connect for Verifiable Presentations Fluss}
    \end{figure}
\end{frame}

\section{Sicherheitsüberlegungen von OIDC4VP}

\begin{frame}{Replay attack}
    \begin{itemize}
        \item Der holder sendet dem verifier ein VP
        \item Ein \textit{Man in the Middle} speichert die Vorstellung
        \item Der \textit{Man in the Middle} kann das gespeicherte VP wiederverwenden
        \item Um dieses Problem zu umgehen, wird eine zufalls Nummer genutzt (challange-response)
    \end{itemize}
\end{frame}

% \begin{frame}{Session fixation attack}
%     \begin{itemize}
%         \item Der Holder legt das VP im Endpoint des Verifiers ab
%         \item Falls das System des Verifiers infisziert ist, kann ein Angreifer das VP einsehen
%         \item Der endpoint des Verifiers retourniert eine redirect URL und ein response code
%         \item Dieser code wird an das Terminal weitergeleitet, somit kann das infiszierte System das VP nicht einsehen
%     \end{itemize}
% \end{frame}

\section{Fazit}

\begin{frame}{Fazit}
    \begin{itemize}
        \item Was hat BBS für Vorteile?
        \begin{itemize}
            \item Selective-disclosure und unlinkability
        \end{itemize}
        \item Wie funktioneren VCs/VPs?
        \item Wie kann man VCs/VPs und BBS verbinden?
        \begin{itemize}
            \item Kanonisierung durch RDF Algorithmus
        \end{itemize}
        \item Wie werden die VPs von holder zu verifier gesendet?
        \begin{itemize}
            \item OpenID connect for Verifiable Presentations
        \end{itemize}
    \end{itemize}
\end{frame}

\begin{frame}{Fazit}
    \centering
    \textbf{\huge Es funktioniert!}
\end{frame}

\section{Ausblick}

\begin{frame}{Ausblick}
    \begin{itemize}
        \item Link Secrets und Blind BBS Signatures für linkability und selective disclosure analysieren
        \item Implementieren und testen
    \end{itemize}
\end{frame}

\end{document}